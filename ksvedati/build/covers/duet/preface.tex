\documentclass[11pt]{article}
\usepackage{fontspec}
\usepackage[utf8]{inputenc}
\setmainfont{Bell MT}
\usepackage[paperwidth=11in,paperheight=8.5in,margin=1in,headheight=0.0in,footskip=0.5in,includehead,includefoot,portrait]{geometry}
\usepackage[absolute]{textpos}
\TPGrid[0.5in, 0.25in]{23}{24}
\parindent=0pt
\parskip=12pt
\usepackage{nopageno}
\usepackage{polyglossia}
\setmainlanguage{english}
\usepackage{graphicx}
\graphicspath{ {./images/} }
\usepackage{amsmath}
\usepackage{multicol}
\usepackage{tikz}
\newcommand*\circled[1]{\tikz[baseline=(char.base)]{
            \node[shape=circle,draw,inner sep=1pt] (char) {#1};}}

\begin{document}

\begin{textblock}{23}(0, 1)
\begin{center}
\huge FOREWORD
\end{center}
\end{textblock}

%\vspace*{0.25\baselineskip}

\begingroup
\begin{center}
\leftskip0.25in
\textit{ksvedati} is a Sanskrit word that means rustling or murmuring.
\rightskip\leftskip
\phantom{text} \hfill

\endgroup

\vspace*{0.25\baselineskip}

\begin{center}
\huge PERFORMANCE NOTES
\end{center}
\begingroup
\begin{center}

\leftskip0.25in
\pmb{String Contact Points} : The indications of string contact positions such as $sul \ tasto$ (abbreviated as $T$), $sul \ ponticello$ (abbreviated as $P$), $extreme \ sul \ tasto$ (abbreviated as $XT$), etc. should be considered as points along the continuum of the length string. The performer should make an effort to smoothly transition from one position to the next throughout the duration of the passage covered by the arrow-demarcated dashed line. When this arrow is not present, the performer should default to an $ordinario$ position. Because the instrumental preparations are near the bridge, all indications of sul tasto and sul ponticello should be considered to be shifted up the length of the string so that sul ponticello is near the highest preparation.
\rightskip\leftskip
\phantom{text} \hfill \phantom{()}


\pmb{Preparations} : Each instrument is uniquely prepared, but a shared common element is the addition of weights to each string in order to distort the overtones of the strings. This piece was composed with circular paperclips in mind, but the effect can be reproduced with Blu Tack of the same weight distributed over the same area of the string. In fact, Blu Tack may be more stable, as paperclips have a tendency to drift off the strings. \circled{1} The Violin features a traditional bow and it also makes use of a specialized ``guiro'' bow where fishing line has been wrapped around the bow hair and legno. It should attach a carbon fiber ``sign clip'' rod to the bridge. The teeth of the alligator clamp may leave a slight indentation on the wood of the bridge, but it is possible to add a protective layer if desired, while slightly muffling the desired sound effect. \circled{2} The Violoncello will be asked to bow directly on a wooden practice mute with both hair and legno to produce a tone. In absence of a wooden mute, bowing on the tailpiece is an acceptable substitute.
\rightskip\leftskip
\phantom{text} \hfill \phantom{()}

\pagebreak

\leftskip0.25in
\includegraphics[height=5cm]{./images/violin_2_prep.png} Violin preparations.
\rightskip\leftskip
\phantom{text} \hfill \phantom{()}


\vspace*{0.25cm}


\leftskip0.25in
\includegraphics[height=5cm]{./images/guiro_bow.png} Guiro Bow.
\rightskip\leftskip
\phantom{text} \hfill \phantom{()}

\vspace*{0.25cm}

\leftskip0.25in
\includegraphics[height=5cm]{./images/cello_prep.png} Violoncello preparations.
\rightskip\leftskip
\phantom{text} \hfill \phantom{()}

\end{multicols}


\leftskip0.25in
\pmb{Miscellaneous} : \circled{1} Tremoli should accelerate and decelerate as indicated by the black bars above the staff. \circled{2} Diamond note heads represent a left hand finger pressure of a natural harmonic. \circled{3} Half-harmonic finger pressure is shown with a diamond half-filled with black for short durations and a diamond open on one end for long durations. \circled{4} When the Violin is meant to bow on the spike attached to the bridge, the position is indicated graphically.
\rightskip\leftskip
\phantom{text} \hfill \phantom{()}

\end{center}
\endgroup

\vspace*{9\baselineskip}


\vspace*{10\baselineskip}

\begin{center}
duration: c. 6'
\end{center}

\end{document}
